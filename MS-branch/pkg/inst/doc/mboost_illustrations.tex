
\documentclass{article}
\usepackage{amstext}
\usepackage{amsfonts}
\usepackage{hyperref}
\usepackage[round]{natbib}
\usepackage{hyperref}
\usepackage{graphicx}
\usepackage{rotating}
%%\usepackage[nolists]{endfloat}
%%\usepackage{Sweave}

%%\VignetteIndexEntry{mboost Illustrations}

\newcommand{\Rpackage}[1]{{\normalfont\fontseries{b}\selectfont #1}}
\newcommand{\Robject}[1]{\texttt{#1}}
\newcommand{\Rclass}[1]{\textit{#1}}
\newcommand{\Rcmd}[1]{\texttt{#1}}
\newcommand{\Roperator}[1]{\texttt{#1}}
\newcommand{\Rarg}[1]{\texttt{#1}}
\newcommand{\Rlevel}[1]{\texttt{#1}}

\newcommand{\RR}{\textsf{R}}
\renewcommand{\S}{\textsf{S}}
\newcommand{\df}{\mbox{df}}

\RequirePackage[T1]{fontenc}
\RequirePackage{graphicx,ae,fancyvrb}
\IfFileExists{upquote.sty}{\RequirePackage{upquote}}{}
\usepackage{relsize}

\DefineVerbatimEnvironment{Sinput}{Verbatim}{baselinestretch=1}
\DefineVerbatimEnvironment{Soutput}{Verbatim}{fontfamily=courier,
                                              baselinestretch=1,
                                              fontshape=it,
                                              fontsize=\relsize{-1}}
\DefineVerbatimEnvironment{Scode}{Verbatim}{}
\newenvironment{Schunk}{}{} 

\renewcommand{\baselinestretch}{1}



\hypersetup{%
  pdftitle = {mboost Illustrations},
  pdfsubject = {package vignette},
  pdfauthor = {Torsten Hothorn and Peter Buhlmann},
%% change colorlinks to false for pretty printing
  colorlinks = {true},
  linkcolor = {blue},
  citecolor = {blue},
  urlcolor = {red},
  hyperindex = {true},
  linktocpage = {true},
}

\begin{document}

\setkeys{Gin}{width=\textwidth}

\title{\Rpackage{mboost} Illustrations}

\author{Torsten Hothorn$^1$ and Peter B\"uhlmann$^2$}

\date{}

\maketitle

\noindent$^1$ Institut f\"ur Medizininformatik, Biometrie und Epidemiologie\\
           Friedrich-Alexander-Universit\"at Erlangen-N\"urnberg\\
           Waldstra{\ss}e 6, D-91054 Erlangen, Germany \\
           \texttt{Torsten.Hothorn@R-project.org}
           \newline

\noindent$^2$ Seminar f{\"u}r Statistik \\ 
              ETH Z\"urich, CH-8092 Z{\"u}rich, Switzerland \\
              \texttt{buhlmann@stat.math.ethz.ch}
           \newline

\section{Illustrations}

This document reproduces the data analyses presented in
\cite{BuhlmannHothorn06}. For a description of the theory behind
applications shown here we refer to the original manuscript.
Note: The \textit{Breast Cancer Subtypes} example is missing from this
document because we cannot assume package \Rpackage{Biobase} to be
installed.

 

\paragraph{Illustration: Prediction of total body fat}

	%%%% DON'T EDIT THIS FILE

\setkeys{Gin}{width = 0.95\textwidth}
\citet{garcia2005} report on the development of predictive regression equations
for body fat content by means of $p = 9$ common anthropometric
measurements which were obtained for $n = 71$ healthy German women. 
In addition, the women's body composition was measured by 
Dual Energy X-Ray Absorptiometry (DXA). This reference method 
is very accurate in measuring body fat but finds little applicability
in practical environments, mainly because of high costs and the 
methodological efforts needed. Therefore, a simple regression equation 
for predicting DXA measurements of body fat is of special interest for the practitioner. 
Backward-elimination was applied to select
important variables from the available anthropometrical measurements and
\citet{garcia2005} report a final linear model utilizing
hip circumference, knee breadth and a compound covariate which is defined as
the sum of log chin skinfold, log triceps skinfold and log subscapular skinfold:
\begin{Schunk}
\begin{Sinput}
R> bf_lm <- lm(DEXfat ~ hipcirc + kneebreadth + anthro3a, 
         data = bodyfat)
R> coef(bf_lm)
\end{Sinput}
\begin{Soutput}
(Intercept)     hipcirc kneebreadth    anthro3a 
  -75.23478     0.51153     1.90199     8.90964 
\end{Soutput}
\end{Schunk}
A simple regression formula which is easy to communicate, such as a 
linear combination of only a few covariates, is of special interest in this 
application: we employ the \Rcmd{glmboost} function from package 
\Rpackage{mboost}
to fit a linear regression model by means of $L_2$Boosting with componentwise
linear least squares. By default, the function \Rcmd{glmboost} fits a linear model (with 
initial $m_\text{stop} = 100$ and shrinkage parameter $\nu = 0.1$) by
minimizing squared error (argument \Rcmd{family = GaussReg()} is 
the default): 
\begin{Schunk}
\begin{Sinput}
R> bf_glm <- glmboost(DEXfat ~ ., data = bodyfat, 
         control = boost_control(center = TRUE))
\end{Sinput}
\end{Schunk}
Note that, by default, the mean of the response variable is used as an
offset in the first step of the boosting algorithm. We center the covariates
prior to model fitting in addition.
As mentioned above, the special form of the base learner,
i.e., componentwise linear least squares, 
allows for a reformulation of the boosting fit in terms of a linear combination
of the covariates which can be assessed via
\begin{Schunk}
\begin{Sinput}
R> coef(bf_glm)
\end{Sinput}
\begin{Soutput}
 (Intercept)          age    waistcirc      hipcirc 
    0.000000     0.013602     0.189716     0.351626 
elbowbreadth  kneebreadth     anthro3a     anthro3b 
   -0.384140     1.736589     3.326860     3.656524 
    anthro3c      anthro4 
    0.595363     0.000000 
attr(,"offset")
[1] 30.783
\end{Soutput}
\end{Schunk}

\setkeys{Gin}{height=0.9\textheight, keepaspectratio}
\begin{figure}
\begin{center}
\includegraphics{figures/bodyfat_glmboost-bodyfat-oob-plot.pdf}
\caption{\Robject{bodyfat} data: Out-of-bootstrap squared error for varying
  number of  
         boosting iterations $m_\text{stop}$ (top). The dashed horizontal line
         depicts the average out-of-bootstrap error of the linear model 
         for the pre-selected variables \Robject{hipcirc},
         \Robject{kneebreadth}  
         and \Robject{anthro3a} fitted via ordinary least squares. 
         The lower part shows the corrected AIC criterion.
         \label{bodyfat-oob-plot}}
\end{center}
\end{figure}


We notice that most covariates have been used for fitting
and thus no extensive variable selection was performed in the above model. 
Thus, we need to investigate how many boosting iterations are appropriate. Resampling
methods such as cross-validation or the bootstrap can be used to estimate
the out-of-sample error for a varying number of boosting iterations. The
out-of-bootstrap mean  
squared error for $100$ bootstrap samples is depicted in the upper part of 
Figure~\ref{bodyfat-oob-plot}. The plot
leads to the impression that approximately $m_\text{stop} = 44$ would be a sufficient
number of boosting iterations.
In Section~\ref{subsec.stopping}, a corrected version of the Akaike
information criterion (AIC) is proposed for determining the optimal number
of boosting iterations. This criterion attains its 
minimum for
\begin{Schunk}
\begin{Sinput}
R> mstop(aic <- AIC(bf_glm))
\end{Sinput}
\begin{Soutput}
[1] 45
\end{Soutput}
\end{Schunk}
boosting iterations, see the bottom part of
Figure~\ref{bodyfat-oob-plot} in addition.
The coef\-ficients of the linear model with 
$m_\text{stop} = 45$ 
boosting iterations are
\begin{Schunk}
\begin{Sinput}
R> coef(bf_glm[mstop(aic)])
\end{Sinput}
\begin{Soutput}
 (Intercept)          age    waistcirc      hipcirc 
   0.0000000    0.0023271    0.1893046    0.3488781 
elbowbreadth  kneebreadth     anthro3a     anthro3b 
   0.0000000    1.5217686    3.3268603    3.6051548 
    anthro3c      anthro4 
   0.5043133    0.0000000 
attr(,"offset")
[1] 30.783
\end{Soutput}
\end{Schunk}
and thus 7 covariates have been selected for the final
model (intercept equal to zero occurs here for mean centered response and
predictors and hence, 
%TORSTEN: PLEASE CHECK
$n^{-1} \sum_{i=1}^n Y_i = 30.783$
%TORSTEN
is the intercept in the uncentered model). Note that    
the variables \Robject{hipcirc}, \Robject{kneebreadth} and
\Robject{anthro3a}, which 
we have used for fitting a linear model at the beginning of this
paragraph, have been selected by the boosting algorithm as well.  


 

\paragraph{Illustration: Prediction of total body fat (cont.)}

	%%%% DON'T EDIT THIS FILE

\setkeys{Gin}{width = 0.95\textwidth}
Being more flexible than the linear model which we fitted to the
\Robject{bodyfat} data in Section~\ref{subsec.compLS}, we estimate an additive
model 
%for the mean-centered covariates 
using the \Rcmd{gamboost} function from \Rpackage{mboost} (first with
pre-specified $m_\text{stop} = 100$ 
boosting iterations, $\nu = 0.1$ and squared error loss):
\begin{Schunk}
\begin{Sinput}
R> bf_gam <- gamboost(DEXfat ~ ., data = bodyfat)
\end{Sinput}
\end{Schunk}
The degrees of freedom in the componentwise smoothing spline base procedure
can be defined by the \Rcmd{dfbase} argument, defaulting to $4$.  

We can estimate the number of boosting iterations $m_\text{stop}$
using the corrected AIC criterion described in
Section~\ref{subsec.stopping}
%(see Figure~\ref{bodyfat-gamboost-AIC}) 
via
\begin{Schunk}
\begin{Sinput}
R> mstop(aic <- AIC(bf_gam))
\end{Sinput}
\begin{Soutput}
[1] 46
\end{Soutput}
\end{Schunk}
Similar to the linear regression model, the partial contributions of the covariates
can be extracted from the boosting fit. For the most important variables, 
the partial fits are given in Figure~\ref{bodyfat-gamboost-plot} showing some 
slight non-linearity, mainly for \Robject{kneebreadth}. 

%\begin{figure}
%\begin{center}
%<<bodyfat-gamboost-AIC, echo = FALSE, fig = TRUE, height = 4>>= 
%plot(aic, ylim = c(3, 5.5))
%@
%\caption{\Robject{bodyfat} data: Corrected AIC as a function of the number of 
%boosting iterations $m_\text{stop}$ 
%for fitting an additive model via \Rmd{gamboost}. \label{bodyfat-gamboost-AIC}}
%\end{center}
%\end{figure}

\setkeys{Gin}{width=0.95\textwidth, keepaspectratio}
\begin{figure}[t]
\begin{center}
\includegraphics{figures/BH-bodyfat-gamboost-plot}
\caption{\Robject{bodyfat} data: Partial contributions of four covariates
  in an additive model (without centering of estimated functions to mean
  zero).    
         \label{bodyfat-gamboost-plot}}
\end{center}
\end{figure}

 

\paragraph{Illustration: Prediction of total body fat (cont.)}

	%%%% DON'T EDIT THIS FILE

\setkeys{Gin}{width = 0.95\textwidth}
Such transformations and estimation of a corresponding linear model can be
done with the \Rcmd{glmboost} function,
%%\citep[an implementation of the original fractional polynomials approach is
%%available in package \Rpackage{mfp}, see][]{pkg:mfp,Sauerbreietal2006},
where the model formula performs the computations of all transformations by 
means of the \Rcmd{bs} (B-spline basis) function from the package
\Rpackage{splines}.  
First, we set up a formula transforming each covariate 
\begin{Schunk}
\begin{Sinput}
R> bsfm
\end{Sinput}
\begin{Soutput}
DEXfat ~ bs(age) + bs(waistcirc) + bs(hipcirc) + bs(elbowbreadth) + 
    bs(kneebreadth) + bs(anthro3a) + bs(anthro3b) + bs(anthro3c) + 
    bs(anthro4)
\end{Soutput}
\end{Schunk}
and then fit the complex linear model by using the \Rcmd{glmboost} function with 
initial $m_\text{stop} = 5000$ boosting iterations:
\begin{Schunk}
\begin{Sinput}
R> ctrl <- boost_control(mstop = 5000)
R> bf_bs <- glmboost(bsfm, data = bodyfat, control = ctrl)
R> mstop(aic <- AIC(bf_bs))
\end{Sinput}
\begin{Soutput}
[1] 2891
\end{Soutput}
\end{Schunk}
The corrected AIC criterion (see Section~\ref{subsec.stopping}) suggests to
stop after $m_\text{stop} = 2891$ 
boosting iterations and the final model selects
$21$ (transformed) predictor
variables. Again, the partial 
contributions of each of the $9$ original covariates
can be computed 
easily and are shown in Figure~\ref{bodyfat-fpboost-plot} (for the same
variables as in Figure~\ref{bodyfat-gamboost-plot}). Note that the depicted
functional relationship derived from the model fitted above
(Figure~\ref{bodyfat-fpboost-plot}) is qualitatively the same as the one
derived from the additive model (Figure~\ref{bodyfat-gamboost-plot}).
%, which is computationally more demanding.

\setkeys{Gin}{width=0.95\textwidth, keepaspectratio} 
\begin{figure}[t]
\begin{center}
\includegraphics{figures/BH-bodyfat-fpboost-plot}
\caption{\Robject{bodyfat} data: Partial fits for a linear model fitted to
  transformed covariates using B-splines (without centering of estimated
  functions to mean zero). \label{bodyfat-fpboost-plot}}
\end{center}
\end{figure}



\paragraph{Illustration: Wisconsin prognostic breast cancer}

	%%%% DON'T EDIT THIS FILE

\setkeys{Gin}{width = 0.95\textwidth}
Prediction models for recurrence events in breast cancer patients 
based on covariates which have been computed from a digitized image of a
fine needle aspirate of breast tissue (those measurements describe
characteristics of the cell nuclei present in the image) have been studied
by \citet{street1995} \citep[the data is part of the UCI repository][]{uci1998}.

We first analyze this data as a binary prediction problem 
(recurrence vs. non-recurrence) and later in Section~\ref{sec.survival} 
by means of survival models. We are faced with many covariates ($p = 32$) 
for a limited number of observations without missing values 
($n = 194$), and variable selection is an important issue. We can
choose a classical logistic regression model  
via AIC in a stepwise algorithm as follows 
%(after centering the covariates)
\begin{Schunk}
\begin{Sinput}
R> cc <- complete.cases(wpbc)
R> wpbc2 <- wpbc[cc, colnames(wpbc) != "time"]
R> wpbc_step <- step(glm(status ~ ., data = wpbc2, 
         family = binomial()), trace = 0)
\end{Sinput}
\end{Schunk}
The final model consists of 16 parameters with
\begin{Schunk}
\begin{Sinput}
R> logLik(wpbc_step)
\end{Sinput}
\begin{Soutput}
'log Lik.' -80.13 (df=16)
\end{Soutput}
\begin{Sinput}
R> AIC(wpbc_step)
\end{Sinput}
\begin{Soutput}
[1] 192.26
\end{Soutput}
\end{Schunk}
and we want to compare this model to a logistic regression model fitted via gradient boosting. 
We simply select the \Robject{Binomial} family (with default offset of $1/2
\log(\hat{p} / (1 - \hat{p}))$,  
where $\hat{p}$ is the empirical proportion of recurrences) and we
initially use
$m_\text{stop} = 500$ boosting iterations 
\begin{Schunk}
\begin{Sinput}
R> ctrl <- boost_control(mstop = 500, center = TRUE)
R> wpbc_glm <- glmboost(status ~ ., data = wpbc2, 
         family = Binomial(), control = ctrl)
\end{Sinput}
\end{Schunk}
%The negative binomial log-likelihood is
%<<wpbc-glmboost-loglik, echo = TRUE>>= 
%logLik(wpbc_glm)
%@
The classical AIC criterion (-2 log-likelihood + 2 $\df$) suggests to stop after
\begin{Schunk}
\begin{Sinput}
R> aic <- AIC(wpbc_glm, "classical")
R> aic
\end{Sinput}
\begin{Soutput}
[1] 199.54
Optimal number of boosting iterations: 465 
Degrees of freedom (for mstop = 465): 9.147 
\end{Soutput}
\end{Schunk}
boosting iterations. 
%%Again, we want to investigate the out-of-bootstrap performance
%%of the boosted logistic regression model. Figure~\ref{wpbc-benchmarks-plot} depicts the
%%out-of-bootstrap (positive) binomial log-likelihood, standardized with 
%%the number of out-of-bootstrap observations, for varying number of boosting iterations $m_\text{stop}$. 
%%\begin{figure}
%%\begin{center}
%%<<wpbc-benchmarks-plot, echo = FALSE, fig = TRUE>>=
%%load(system.file("cache/wpbc_benchmarks.rda", package = "mboost"))
%%grid <- seq(from = 5, to = 500, by = 5)
%%plot(c(0, grid[-1] - 5), -colMeans(boob), type = "b", 
%%     ylab = "Log-Likelihood / n",
%%     xlab = "Number of Boosting Iterations")
%%mopt <- grid[which.max(colMeans(boob))]
%%@
%%\caption{\Robject{wpbc} data: Out-of-bootstrap (positive) binomial 
%%    log-likelihood. \label{wpbc-benchmarks-plot}}
%%\end{center}
%%\end{figure}
We now restrict the number of boosting iterations to 
$m_\text{stop} = 465$ and then obtain the estimated
coefficients via 
\begin{Schunk}
\begin{Sinput}
R> wpbc_glm <- wpbc_glm[mstop(aic)]
R> coef(wpbc_glm)[abs(coef(wpbc_glm)) > 0]
\end{Sinput}
\begin{Soutput}
      (Intercept)       mean_radius      mean_texture 
      -1.2511e-01       -5.8453e-03       -2.4505e-02 
  mean_smoothness     mean_symmetry   mean_fractaldim 
       2.8513e+00       -3.9307e+00       -2.8253e+01 
       SE_texture      SE_perimeter    SE_compactness 
      -8.7553e-02        5.4917e-02        1.1463e+01 
     SE_concavity  SE_concavepoints       SE_symmetry 
      -6.9238e+00       -2.0454e+01        5.2125e+00 
    SE_fractaldim      worst_radius   worst_perimeter 
       5.2187e+00        1.3468e-02        1.2108e-03 
       worst_area  worst_smoothness worst_compactness 
       1.8646e-04        9.9560e+00       -1.9469e-01 
            tsize            pnodes 
       4.1561e-02        2.4445e-02 
\end{Soutput}
\end{Schunk}
(because of using the offset-value $\hat{f}^{[0]}$, we have to add the value
$\hat{f}^{[0]}$ to the reported intercept estimate above for the logistic
regression model). 
%%We then can
%%extract the fitted conditional probabilities  
%%<<wpbc-glmboost-prob, echo = TRUE>>=
%%f <- fitted(wpbc_glm)
%%### probabilities
%%p <- exp(f) / (exp(f) + exp(-f))
%%@
%%which are depicted by a conditional density plot in 
%%Figure~\ref{wpbc-glmboost-prob-plot}.

%\setkeys{Gin}{width=0.7\textwidth, keepaspectratio} 
%\begin{figure}
%\begin{center}
%<<wpbc-glmboost-prob-plot, echo = FALSE, fig = TRUE>>=
%cdplot(status ~ p, data = wpbc2) 
%@
%\caption{\Robject{wpbc} data: Conditional density plot of the fitted 
%         probabilities of recurrence / non-recurrence. \label{wpbc-glmboost-prob-plot}}
%\end{center}
%\end{figure}

A generalized additive model adds more flexibility to the regression function but is still
interpretable. We fit a logistic additive model to the \Robject{wpbc} data as follows:
\begin{Schunk}
\begin{Sinput}
R> wpbc_gam <- gamboost(status ~ ., data = wpbc2, 
         family = Binomial())
R> mopt <- mstop(aic <- AIC(wpbc_gam, "classical"))
R> aic
\end{Sinput}
\begin{Soutput}
[1] 196.33
Optimal number of boosting iterations: 84 
Degrees of freedom (for mstop = 84): 13.754 
\end{Soutput}
\end{Schunk}
This model selected $16$ out of $32$
covariates. The partial contributions of the four most important variables
are depicted in  
Figure~\ref{wpbc-gamboost-plot} indicating a remarkable degree of non-linearity.
\setkeys{Gin}{width=0.95\textwidth, keepaspectratio} 
\begin{figure}[t]
\begin{center}
\includegraphics{figures/BH-wpbc-gamboost-plot}
\caption{\Robject{wpbc} data: Partial contributions of four selected 
    covariates in an additive logistic model (without centering of
    estimated functions to mean zero).
    \label{wpbc-gamboost-plot}}
\end{center}
\end{figure}



\paragraph{Illustration: Wisconsin prognostic breast cancer (cont.)}         

	%%%% DON'T EDIT THIS FILE

\setkeys{Gin}{width = 0.95\textwidth}Instead of the binary response variable describing the recurrence status, we
make use of the additionally available time information for modeling
the time to recurrence, i.e., all observations with non-recurrence are censored.
First, we calculate IPC weights 
\begin{Schunk}
\begin{Sinput}
R> censored <- wpbc$status == "R"
R> iw <- IPCweights(Surv(wpbc$time, censored))
R> wpbc3 <- wpbc[, names(wpbc) != "status"]
\end{Sinput}
\end{Schunk}
and fit a weighted linear model by boosting with componentwise linear
weighted least squares as base procedure: 
\begin{Schunk}
\begin{Sinput}
R> ctrl <- boost_control(mstop = 500, center = TRUE)
R> wpbc_surv <- glmboost(log(time) ~ ., data = wpbc3, 
         control = ctrl, weights = iw)
R> mstop(aic <- AIC(wpbc_surv))
\end{Sinput}
\begin{Soutput}
[1] 122
\end{Soutput}
\begin{Sinput}
R> wpbc_surv <- wpbc_surv[mstop(aic)]
\end{Sinput}
\end{Schunk}
The following variables have been selected for fitting 
\begin{Schunk}
\begin{Sinput}
R> names(coef(wpbc_surv)[abs(coef(wpbc_surv)) > 0])
\end{Sinput}
\begin{Soutput}
 [1] "mean_radius"         "mean_texture"       
 [3] "mean_perimeter"      "mean_smoothness"    
 [5] "mean_symmetry"       "SE_texture"         
 [7] "SE_smoothness"       "SE_concavepoints"   
 [9] "SE_symmetry"         "worst_concavepoints"
\end{Soutput}
\end{Schunk}
and the fitted values are 
depicted in Figure~\ref{wpbc-glmboost-censored-fit}, showing a reasonable model fit.

\setkeys{Gin}{width=0.7\textwidth, keepaspectratio} 
\begin{figure}[t]
\begin{center}
\includegraphics{figures/BH-wpbc-glmboost-censored-fit}
\caption{\Robject{wpbc} data: Fitted values of an IPC-weighted linear
  model, taking both time to recurrence and 
    censoring information into account. The radius of the circles is proportional to the
    IPC weight of the corresponding observation, censored observations with IPC weight zero
    are not plotted. \label{wpbc-glmboost-censored-fit}}
\end{center}
\end{figure}

Alternatively, a Cox model with linear predictor can be fitted using
$L_2$Boosting by 
implementing the negative gradient  
of the partial likelihood (see \cite{ridgew99}) via
%%<<wpbc-CoxPH, echo = TRUE, eval = FALSE>>=
%%ctrl <- boost_control(center = TRUE)
%%glmboost(Surv(wpbc$time, wpbc$status == "N") ~ ., 
%%    data = wpbc, family = CoxPH(), 
%%        control = ctrl)
%%@
\begin{Schunk}
\begin{Sinput}
R> ctrl <- boost_control(center = TRUE)
R> glmboost(Surv(wpbc$time, wpbc$status == "N") ~ ., 
         data = wpbc, family = CoxPH(), control = ctrl)
\end{Sinput}
\end{Schunk}
For more examples, such as fitting an additive Cox model using
\Rpackage{mboost}, see \citep{Hothorn:2006:Bioinformatics:16940323}.  

\clearpage

\bibliographystyle{plainnat}
\bibliography{boost}

\end{document}
